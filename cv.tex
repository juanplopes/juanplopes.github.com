\documentclass[a4paper,12pt,oneside]{article}
\usepackage{units}
\usepackage{hyperref}
\usepackage{xcolor}
\usepackage[utf8]{inputenc} 
\usepackage{a4wide}

\pagenumbering{gobble}

\hypersetup{%
  colorlinks=true,% hyperlinks will be black
  linkcolor=blue,% hyperlink borders will be red
  urlcolor=blue,% hyperlink borders will be red
  linkbordercolor=blue,% hyperlink borders will be red
  pdfborderstyle={/S/U/W 1}% border style will be underline of width 1pt
}

\begin{document}
\author{Juan Lopes}

\begin{center}
	{\huge Hi, I'm Juan Lopes} \\[0.5cm]

	{\large \today}
\end{center}


\renewcommand{\abstractname}{}
\begin{abstract}
\noindent \begin{center}
Brazilian, lives in Rio de Janeiro. \\
Interested in algorithms, optimization, programming contests and serif typefaces. \\
B.S. in Computer Science; M.S. candidate in Computer Science. \\
Writes high-performance Java code at \textit{Intelie} for the last 2 \nicefrac{1}{2} years. \\
Wrote beaufitul C\# code at \textit{Living Consultoria} for almost 4 years before.
\end{center}
\end{abstract}


\section*{What excites me?}

	I love code in all its forms. Love writing and love reading. I'm a programmer since I was 9, when discovered that QBasic used to come with Windows 95. 
	
	During the undergrad years, I had the chance to expand my interest in \textbf{algorithms}. There I was introduced to \textbf{programming contests}. I attended to six editions of \textbf{ICPC} (2007 to 2012), being \textbf{south-american finalist} in five of them. Also in the 2013 edition of the \textbf{IEEEXtreme} contest, my team and I won the \textbf{1st place in Brazil} [\ref{url:ieeextreme}].
	
	I like \textbf{Coding Dojos} a lot. Here in Rio, I help to organize the DojoRio [\ref{url:dojorio}] meetings. We probably have the most active Brazilian dojo community. As of today, there were \textbf{almost 250 sessions} (once a week, every wednesday). I think Coding Dojos are a great oportunity to try new languages, new problems and talk to great programmers.
	
	Recently, as a part of my work at \emph{Intelie}, I've been studying (and coding) a lot about \textbf{time-series analysis} and \textbf{real-time optimized algorithms}. As a result, I've been giving talks about it, from which I highlight the latest about \textbf{probabilistic data structures} at QCon São Paulo 2013 [\ref{url:qconsp2013}].
	
	Last, but not least, as every passionate programmer, I write code for fun in my free time. You can see most of the recent works in my \textbf{GitHub} [\href{http://github.com/juanplopes}{juanplopes}]. Actually, it says more about me than any of these words.
	
\section*{Where did I work?}

\subsection*{Intelie \small{ (August 2011 -- Now)}}

	I currently work at \textbf{Intelie}, a company focused in building tools for operation intelligence. There I write a lot of \textbf{Java} code. As a former C\# programmer, I'm surprisingly liking it.
	
	Most systems I write today use heavy backend algorithms to analyze real-time data streams at a rate of a \textbf{few terabytes per day}. Most of it would be impossible if we didn't use \textbf{distributed algorithms}. 
	
	There I also had a very particular experience with \textbf{Lucene}, adapting it's core to support indexing tens of billions of log messages per Lucene index, with sub-second distributed query, for the largest brazilian media group.

\subsection*{Living Consultoria \small{ (December 2007 -- July 2011)}}

	At \textbf{Living Consultoria} I was a mix of \textbf{lead developer} and \textbf{architect}. The focus of the company was Microsoft technologies, so I spent most of my time writing \textbf{C\#}, but I also worked with \textbf{Ruby} (on and off Rails) and some \textbf{PHP}. There I played a major role implementing the first projects using \textbf{Scrum} in the company (back in 2008, when it was still cool); participated in some \textbf{international projects}, collaborating with teams from many countries. Traveled to Amsterdam a couple of times, because of this.

\section*{Where did I study?}

	Today I'm a \textbf{Master's candidate} in \textbf{Computer Science} at \textbf{UERJ} (en: \emph{Rio de Janeiro State University}). It is one of the largest Brazilian universities. Given my current expecience, my dissertation will probably be abou \textbf{distributed stream processing algorithms}.

	I also have acquired my Bachelor's degree in Computer Science at \textbf{UERJ} also. Finished in 2013. Actually I finished the regular course in 2009, but stayed there to be able to compete in the ICPC. I graduated with an essay about \textbf{polynomial-time regular expressions implementations} [\ref{url:pyrex}].

\section*{How to find me?}

	As a true geek, I like smartphones. Android, mostly. So I will almost always have my phone with me, You can call me at \textbf{(+55) 21 99317 4772}.
	
	Probably the best way to find me is my e-mail [\href{mailto:me@juanlopes.net}{me@juanlopes.net}]. I'm also pretty active on Twitter [\href{http://twitter.com/juanplopes}{@juanplopes}].

\section*{Links}

	Some links are in portuguese.

\begin{enumerate}

  \item \label{url:ieeextreme} \href{http://www.ieee.org/membership_services/membership/students/competitions/xtreme/xtreme7_final_rankings-country.pdf}{IEEEXtreme 7.0 results by country}
  \item \label{url:dojorio} \href{http://dojorio.org/}{DojoRio.org}
  \item \label{url:qconsp2013} \href{http://www.infoq.com/br/presentations/analisando-fluxo-dados-tempo-real}{QCon São Paulo 2013: Analyzing and Reducing Big Data Streams in Real-Time}
  \item \label{url:pyrex} \href{http://github.com/juanplopes/pyrex}{github.com/juanplopes/pyrex}
\end{enumerate}

\end{document}
